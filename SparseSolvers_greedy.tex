%\bibliographystyle{ieeetr}
%\bibliographystyle{splncs}
%\bibliographystyle{elsart-harv}
%\bibliographystyle{ipsjunsrt-e}
\bibliographystyle{ieicetr}




\Chapter{スパース解法 ---貪欲法 ---}


次のような$\ell_0$最小化問題の解き方を考えよう.
\begin{equation}
 \min_{\small\vec x} \|\vec x\|_0\quad\mbox{ subject to }\quad \mtr A\vec x=\vec b.
\label{eq:l0 min}
\end{equation}

{\bf 貪欲法}\footnote{欲張り法ともいう.}(greedy algorithm)とは何か?
おつりを作る作業を思い出そう.
なるべく少ない枚数のコインで623円をどのように作るか?
内輪で残額に最も近い額のコインを選ぶことを繰り返せばよい\footnote{%
用意されているコインが1,5,10,50,100,500円玉ならば必ず最適解が得られる.
しかし,貪欲法はどんなコインの組合せ問題も解けるわけではない.
例えば,もし1,6,13円玉の3種類だとしたら,18円を3枚の6円玉で作れるのに,
貪欲法では1枚の13円玉と5枚の1円玉の計6枚が選ばれる.}.

線形結合$\vec b=x_1\vec a^{(1)}+\cdots+x_n\vec a^{(n)}$を思い出そう.
基底ベクトル$\vec a^{(j)}$はコインである.
金額$\vec b$を作るために使うコイン$\vec a^{(j)}$の係数$x_j$だけ非ゼロにする.
選んだコインを表す非ゼロ係数の番号(添え字)の集合を$T$とする.
なるべく少ない枚数のコインで金額$\vec b$を作ろう.
残額$\vec r=\vec b-\mtr A\vec x$に最も似ているコイン$\vec a^{(s)}$を選び,
%残額$\vec r=\vec b-\mtr A_T\vec x_T$に最も似ているコイン$\vec a^{(s)}$を選び,
残額が最も減るように非ゼロ成分の$\vec x_{T\cup \{s\}}$を最小二乗法で更新すればよい.


\Section{OMP}

{\bf OMP}(orthogonal matching pursuit)\cite{Pati93,Tropp04,Tropp07}は,
アルゴリズム\ref{alg:OMP}のようにスパース解を求める貪欲法である.

\begin{algorithm}[h]
\caption{Orthogonal matching pursuit: $\vec x = \mbox{\sc OMP}(\vec b,\mtr A,k,\delta)$}
\label{alg:OMP}
\begin{algorithmic}[1]
\REQUIRE
$\vec b$: $m$次元ベクトル,$\mtr A$: $m\times n$行列,
$k$: 非ゼロ成分の最大個数,$\delta$: 許容誤差;
\ENSURE $\vec x$: $k$スパースな$n$次元ベクトル;
\STATE
$\vec x \subs \vec 0\in\mathbb{R}^{n}$;
\STATE
台の初期値を$T\subs\emptyset$(空集合)とする;
\STATE
$\vec r\subs\vec b$;
\WHILE{$|T|\leq k$ かつ $\|\vec r\|_2/\|\vec b\|_2>\delta$}
\STATE
各基底と残差の類似度$\vec c\subs\mtr A^\top\vec r$を計算する;
\label{step:similarity}
\STATE
最も類似している列の番号$s\subs\arg\max_{j} |c_j|$を見つける;
\label{step:find max}
\STATE
$T\subs T\cup\{s\}$として$s$を台に加える;
\label{step:union T}
\STATE
台$T$で示された非ゼロ成分を
$\displaystyle\vec x_T\subs\arg\min_{\vec z\in\mathbb{R}^{|T|}}\|\vec b-\mtr A_T\vec z\|_2^2$のように最小二乗解で更新する.
\label{step:lsq T}
\STATE
残差$\vec r\subs\vec b-\mtr A\vec x$を計算する;
%残差$\vec r\subs\vec b-\mtr A_T\vec x_T$を計算する;
\ENDWHILE
\end{algorithmic}
\end{algorithm}
%Here, $\mtr A_T$ is the $m\times |T|$ matrix composed of the columns
%indicated by $T$.

貪欲法は,ゼロベクトルを解$\vec x$の初期値とし,
残差が十分に小さくなる($\mtr A\vec x$が十分に$\vec b$の近似になる)まで
$\vec x$の非ゼロ成分を次々と見つけるアルゴリズムになっている.
よって,非ゼロ成分の個数程度の反復回数で解が得られる.
各反復において最も計算量が大きい手続きは,
ステップ\ref{step:similarity}の類似度計算($\mathcal{O}(mn)$)と,
ステップ\ref{step:lsq T}の
最小二乗解$\vec x_T$の計算である\footnote{%
この最小二乗解は$\vec x_T=(\mtr A_T^\top\mtr A_T)^{-1}\mtr A_T\vec b$と書ける.
これを右から順に直接に計算する方法の他に,
$\mtr A_T$の特異値分解を利用する方法,
連立方程式$\mtr A_T^\top\mtr A_T\vec x_T=\mtr A_T\vec b$を共役勾配法で解く方法などがある.}.
最小二乗法の手間数は$\mathcal{O}(m|T|^2)$なので,$k^2\ll n$ならば
OMPの手間数は$\mathcal{O}(mk^3)$と見積もられる.


%優決定問題



\Section{演習問題}
\begin{enumerate}
\item
アルゴリズム\ref{alg:OMP}のOMPをMATLAB関数として実装せよ.
\lstset{xleftmargin=4ex,xrightmargin=4ex}
\begin{lstlisting}
function x = OMP(b, A, k, delta)
\end{lstlisting}
この行から書き始め,{\tt OMP.m}として保存せよ.\\
ヒント:
MATLABでは,ステップ\ref{step:find max},\ref{step:union T},
\ref{step:lsq T}は次のように書ける.
\begin{lstlisting}
[maxc, s] = max([abs(c)]);
T = [T, s];
x(T) = A(:,T) \ b;
\end{lstlisting}
% T = union(T, s);

\item \label{ex:random sparse solution}
$\mtr A$,$\vec x$,$\vec b$を以下のように生成し,
$\vec x$を未知として式(\ref{eq:l0 min})をOMPで解け.
\begin{itemize}
\item
$\mtr A$をサイズ$m\times n$のランダム行列とする.
要素は正規分布から生成する.
\item
$\vec x$を$k$スパースなベクトルとし,
非ゼロ成分は正規分布から生成する.
\item $\vec b=\mtr A\vec x$とする.
\end{itemize}
次に,非ゼロ成分の個数$k$に対するOMPの解の誤差を観察せよ.
すなわち,横軸を$k$,
縦軸を相対残差$\|\vec b-\mtr A\vec x\|_2/\|\vec b\|_2$として
グラフを描け.\\
ヒント:
以下のコードを試し,OMPで解が得られるかどうか確認せよ.
\begin{lstlisting}
m = 100; n = 1000; A = randn(m,n);
k = 10; S = randperm(n); S = S(1:k);
x = zeros(n,1); x(S) = randn(k,1);
b = A * x;
x_est = OMP(b, A, k, 1e-7);
figure, plot(1:n, x, '.', 1:n, x_est, 'o');
\end{lstlisting}
このコードを改造して相対残差を観察せよ.
誤差を統計的に評価するため,反復試行する必要がある.

\item\yeeks
\href{http://perception.csl.illinois.edu/recognition/Home.html}{スパース表現による顔認識}\cite{Wright09}\\
\href{http://www.columbia.edu/~jw2966/Wright09-PAMI.pdf}{J. Wright, A. Y. Yang, A. Ganesh, S. S. Sastry, and Y. Ma, ``Robust face recognition via sparse representation,'' IEEE Transactions on Pattern Analysis and Machine Intelligence, vol. 31, no. 2, pp. 210--227, 2009.}\\
について調査し,下記の課題に取り組むこと.
\begin{itemize}
\item ``Sparse Representation-based Classification (SRC)''
と呼ばれる手法の概要を述べよ.
\item 
\href{http://vision.ucsd.edu/~leekc/ExtYaleDatabase/ExtYaleB.html}{The Extended Yale Face Database B}\\
(http://vision.ucsd.edu/~leekc/ExtYaleDatabase/ExtYaleB.html)
をランダム射影したものを用いて,
SRCによる顔認識を実行せよ.
\href{https://drive.google.com/file/d/0Bx4bEpaTSFcgRm15NVlua1pOTWM}{YaleB\_Ext\_Cropped\_192x168\_all.mat}は,
顔を切り出した画像の画素値を並べた行列をMATLABファイルにしたものである.

% (see the code below),
\item (a) ランダム射影後の次元数,(b)ノイズの強さに関して
顔認識の正答率を評価せよ.
\end{itemize}
ヒント:
SRCによる顔認識を実行するコードを以下に示す.
\begin{lstlisting}
% Yale B cropped face imagesを読み込む.
data = load('YaleB_Ext_Cropped_192x168_all.mat');
[n, m] = size(data.fea);

% 訓練データとテストデータに分ける.
ntest = round(n/2); ntrain = n - ntest;
jtest = randperm(n, ntest); jtrain = setdiff(1:n, jtest);
idtest = data.gnd(jtest); idtrain = data.gnd(jtrain);
idmax = max(idtrain);

% データ行列
D = double(data.fea(:,:)');

% ランダム射影(192*168次元から200次元へ)
D = randn(200, 192*168) * D;

% 列ベクトルが正規化された訓練データ行列
A = D(:, jtrain);
A = A * spdiags(1./sqrt(sum(A.^2))', 0, ntrain, ntrain);

%for j = 1:ntest,
j = 700;
    b = D(:, jtest(j)); % + 100*rand(size(y)); % ノイズを印加する場合
    fprintf('testing id = %d\n', idtest(j));
    x = OMP(b, A, 100, 1e-1);
    fprintf('nnz = %d\n', nnz(x));

    % 残差の計算
    residual = zeros(idmax,1);
    for id = 1:idmax,
        b_reconst = A(:,idtrain == id) * x(idtrain == id);
        residual(id) = norm(b - b_reconst);
    end
    figure(11), bar(residual);

    % 識別
    [~, id_est] = min(residual);
    fprintf('estimated id = %d\n', id_est);
%end
\end{lstlisting}



\end{enumerate}




{\footnotesize
\bibliography{mybib_sparse,mybib_cs}
}

